Najważniejszym celem niniejszej pracy było stworzenie języka wysokiego poziomu wspomagającego 
 tworzenie procesów zasilających hurtownie danych.
Język ten odpowiada za częściową automatyzację pracy programisty, najbardziej powtarzalnych i najmniej ciekawych 
czynności, a zatem także najbardziej podatnych na błędy.

Dzięki pracy nad projektem języka wysokiego poziomu udało się osiągnąć zamierzony efekt,
 którym było zminimalizowanie ilości napisanego kodu,
 a tym samym zwiększenie efektywności pracy programisty. 
Osiągnięty cel potwierdzają listingi przykładowych użyć programu,
 które zostały zamieszczone w podrozdziale \ref{p_r_przyklady}. 

Stworzony na potrzeby pracy program posiada następujące cechy:

\begin{itemize} 
\item skrypt odpowiedzialny za pobranie pliku z danymi zostaje utworzony przy użcyiu jednego polecenia,
\item skrypt pgloader'a odpowiedzialny za załadowanie danych do bazy wygenerowany zostaje również przy użyciu jednego polecenia,
\item raz utworzona tabela wymiarów lub tabela faktów jest wielokrotnie wykorzystywana do tworzenia procesów zasilających,
\item automatycznie tworzone przez program tabele przejściowe i tabele docelowe pozwalają uniknąć błędów, które pojawiają
      się często podczas pisania kodu SQL,
\item język, w którym został napisany program jest prosty i intuicyjny, co pozwala uniknąć błędów składniowych.
\item elastyczność działania wynikająca z niezależności względem wykorzystywanego systemu bazodanowego. 
\end{itemize}

Dzięki wielu godzinom poświęconym hurtowniom danych,
 zauważyłem jak niewiele w tej dziedzinie zostało poświęcone odpowiednim procesom
automatyzującym i przyspieszającym tworzenie hurtowni danych.
Dzięki zdobytemu podczas pisania tej pracy doświadczeniu spostrzegłem
 jak istotne jest to zagadnienie dla dziedziny jaką są hurtownie danych i jak procesy usprawniające tworzenie hurtowni mogą wpłynąć
 na efektywność pracy programistów.
Dlatego też za cel obrałem sobie stworzenie języka wysokiego poziomu, który by tą pracę usprawniał, 
a zarazem czynił ją łatwiejszą. Program, który został napisany na potrzeby tej pracy dyplomowej może być wykorzystywany 
jako język generujący hurtownie danych "w locie" albo może po prostu usprawniać pracę programisty zajmującego się hurtowniami danych.

Tworzenie hurtowni danych jest procesem bardzo złożonym, dlatego perspektywa dalszego rozwoju języka daje również szerokie możliwości,
które nie kończą się bynajmniej na funkcjonalnościach opisanych w powyższej pracy.
Dostrzegam wiele potencjalnych funkcjonalności, o które można by rozbudować ten język. Do tych funkcjonalności zaliczam:

\begin{enumerate}
  \item stworzenie skryptu, którego zadaniem byłoby uruchamianie w sposób cykliczny utworzonych procesów zasilających,
  \item dodawanie lub usuwanie nowych elementów w warstwach procesów zasilających,
  \item zapisywanie informacji o utworzonych procesach w bazie danych w sposób umożliwiający 
        stworzenie zapytania zwracającego poprawną kolejność uruchamianych procesów,
  \item poszerzenie składni języka, która mogła by posłużyć do zapisywania
        ważnych informacji dotyczących tabel, dzięki czemu będzie możliwe automatyczne tworzenie częściowej dokumentacji w języku \LaTeX,
  \item utworzone pliki można by przechowywać na dysku w uporządkowanej strukturze katalogów,
  \item uniwersalność języka sprawia, że może być on w przyszłości rozwijany niezależnie od wykorzystywanego systemu bazodanowego.
\end{enumerate}
