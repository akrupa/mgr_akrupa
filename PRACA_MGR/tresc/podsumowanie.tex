Głównym celem niniejszej pracy było utworzenie języka wysokiego poziomu do tworzenia procesów zasilających hurtownie danych, 
 którego zadaniem będzie częsciowa automatyzacja pracy programisty,
 która jest najbardziej powtarzalna,
 a więc najmniej ciekawa i zarazem najbardziej podatna na błędy.

Dzięki pracy nad projektem języka wysokiego poziomu udało się osiągnąć zamierzony efekt,
 który doprowadził do zminimalizowania ilości napisanego kodu, 
 a więc tym samym do zwiększenia produktywności.
Osiągnięty cel potwierdzają przykłady zamieszczone w podrozdziale \ref{p_r_przyklady}.

Program, który został stworzony na potrzeby pracy dyplomowej może być używany jako język generujący
 hurtownie danych "w locie" albo może po prostu usprawnić pracę programisty.

Znajomość zasad programowania C++ pozwala na dodanie nowych etapów przetwarzania lub ich usunięcie,
 natomiast użycie PostgreSQL i Bash nie umniejsza ogólności utworzonego języka wysokiego poziomu
 do tworzenia procesów zasilających hurtownie danych.
