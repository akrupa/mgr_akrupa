Głównym celem niniejszej pracy było utworzenie języka wysokiego poziomu do tworzenia procesów zasilających hurtownie danych,
 który zautomatyzuje część pracy programisty najbardziej powtarzalną, a więc najmniej ciekawą i najbardziej podatną na błędy.

Udało się osiągnąć zamierzony efekt,
 który doprowadził do zmniejszenia ilość kodu do napisania, 
 a więc zwiększenia produktywności,
 świadczą o tym przykłady zamieszczone w podrozdziale \ref{p_r_przyklady}

Program, który został napisany, może być używany jako język generujący hurtownie danych "w locie",
 albo może tylko usprawnić pracę programisty.

Znajomość zasad programowania C++ pozwala na dodanie nowych etapów przetwarzania lub ich usunięcie.
Natomiast użycie Postgessql i Bash nie umniejsza ogólności utworzonego języka wysokiego poziomu
 do tworzenia procesów zasilających hurtownie danych.

