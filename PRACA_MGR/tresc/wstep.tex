
Ojcem koncepcji hurtowni danych jest Bill Inmon, napisał on ponad 40 książek związanych z tą tematyką.
Koncepcja ta dotyczy, jak wspomóc osoby zarządzające firmą, korporacją w podejmowaniu działań strategicznych.
Hurtownie danych odniosły sukces związany z  problemem biznesowym związanym z zarządzaniem relacjami z~klientem,
 w~skrócie CRM, \ang{Customer Relationship Management}


Projektowanie jaki i tworzenie hurtowni danych jest procesem bardzo złożonym i~kosztownych, który trwa od pół roku do dwóch lat.
Firmy podejmujące decyzje o inwestycji utworzenia hurtowni danych, są świadome, 
 że nie produkt zakupiony ma generować zyski tylko dostarczać wiarygodnych i rzetelnych informacji, 
 na~podstawie których możliwe jest podjęcie decyzji strategicznych. 
Jeżeli projekt hurtowni danych jest nie ukierunkowany pod danego klienta lub przechowuje niepoprawne dane, 
 to staje się dużą stratą dla firmy.

Celem pracy jest napisanie języka wysokiego poziomu, 
 który pomoże programistą w tworzeniu procesów zasilających hurtownie danych.
 Zadaniem owego języka, na podstawie podanych poleceń jest wygenerowanie:

\begin{itemize}
 \item szablonu pobierającego dane (źródło),
 \item szablonu pgloader lub gotowe polecenia insert,
 \item kodu umożliwiającego utworzenie tabeli,
 \item kodu języka SQL zasilającego table.
\end{itemize}

Pierwszy rozdział pracy opisuje hurtownie danych i powody jej budowanie,
 jak również  została przedstawiona w nim  architektura hurtowni danych.
Kolejny rozdział opisuje, czym są procesy zasilania hurtowni danych, 
 oraz omawia przykładowe procesy zasilania, które są realizowane w ramach niniejszej pracy.


(
Zastanawiam się jeszcze, czy w tym miejscu napisać o bazie PostgreSQL 8.4.14 ,czy na samym końcu wstępu.
Język wysokiego poziomu,który mam napisać, planuje testować na tej bazie. 
)

 
(zarys 3,4,5)
Trzeci rozdział zawiera definicję i pojęcia, które są niezbędne są do 
zrozumienia czwartego rozdziału pracy, opisującą tematykę tworzenia języków interpretowanych .
Ostatni rozdział niniejszej pracy dotyczy opisu programu wraz z przykładem.
  


