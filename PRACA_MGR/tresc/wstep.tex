
Ojcem koncepcji hurtowni danych jest Bill Inmon,
 napisał on ponad 40 książek związanych z tą tematyką.
Koncepcja ta dotyczy,
 jak wspomóc osoby zarządzające firmą,
 korporacją w podejmowaniu działań strategicznych.
Hurtownie danych odniosły sukces związany z problemami biznesowymi związanymi z zarządzaniem relacjami z~klientem,
 w~skrócie CRM, \ang{Customer Relationship Management}
Projektowanie jaki i tworzenie hurtowni danych jest procesem bardzo złożonym i~kosztownych,
 który trwa od pół roku do dwóch lat.
Firmy podejmujące decyzje o inwestycji utworzenia hurtowni danych,
 są świadome, 
 że nie produkt zakupiony ma generować zyski tylko dostarczać wiarygodnych i rzetelnych informacji, 
 na~podstawie których możliwe jest podjęcie decyzji strategicznych. 
Jeżeli projekt hurtowni danych jest nie ukierunkowany pod danego klienta lub przechowuje niepoprawne dane, 
 to staje się dużą stratą dla firmy.\cite{TodMan}

Celem pracy jest napisanie języka wysokiego poziomu, 
 który pomoże programistom w tworzeniu procesów zasilających hurtownie danych.
 Zadaniem owego języka, na podstawie podanych poleceń jest wygenerowanie:

\begin{itemize}
 \item szablonu pobierającego dane (źródło),
 \item szablonu pgloader lub gotowego polecenia insert,
 \item kodu umożliwiającego utworzenie tabeli,
 \item kodu języka SQL zasilającego tabele.
\end{itemize}

Pierwszy rozdział pracy opisuje hurtownie danych i powody jej budowania,
 jak również została przedstawiona w nim architektura hurtowni danych.
Kolejny rozdział opisuje, czym są procesy zasilania hurtowni danych, 
 oraz omawia przykładowe procesy zasilania, które są realizowane w ramach niniejszej pracy.
Trzeci rozdział niniejszej pracy został natomiast poświęcony teorii związanej 
z~tworzeniem języków interpretowanych.

Ostatni rozdział pracy dotyczy opisu programu wraz z przykładem.
 Wszystkie przykłady zamieszczone w niniejszej były testowane 
w składni języka SQL akceptowanego 
przez PostgreSQL 8.4.14 na systemie ubuntu 10.04 LTS.
Powodami wybrania bazy danych PstgresSQL jest:
\begin{itemize}
 \item bezpłatne oprogramowanie i uznane za dobre do zastosowań komercyjnych,
 \item wykorzystywane obecnie w firmie,
    w której zdobywam doświadczenie zawodowe pracując przy tworzeniu warstwy pośredniej hurtowni bazy danych.
\end{itemize}

Analizator składniowy i leksykalny został utworzony przy użyciu otwartego oprogramowania LEX i YACC, 
 które w znaczący sposób ułatwiają tworzenie pierwszych dwóch etapów tworzenia języków interpretowanych i kompilatorów.
Lex i Yacc są programami,
 które wymagają dużego nakładu pracy,
 by móc dobrze zrozumieć ich działanie i zalety jakich nam dostarczają w tworzeniu języka wysokiego poziomu.
Ponieważ celem niniejszej pracy nie jest tworzenie języków wysokiego poziomu,
 lecz napisanie go i aby nie odbiegać od głównego celu pracy
 opis działania jak również składni zostanie pominięty.

  
