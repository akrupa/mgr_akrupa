Ojcem koncepcji hurtowni danych jest Bill Inmon.
Napisał on ponad 40 książek związanych z tematyką hurtowni danych.
Koncepcja ta tłumaczy,
 w jaki sposób wspomagać osoby zarządzające firmą lub korporacją w podejmowaniu działań strategicznych.
Hurtownie danych odniosły sukces związany w rozwiązywaniu problemów biznesowych w zarządzaniu relacjami z~klientem,
 w~skrócie CRM \ang{Customer Relationship Management}.
Projektowanie jak i tworzenie hurtowni danych jest procesem bardzo złożonym i~kosztownym,
 który trwa od pół roku do dwóch lat. Firmy podejmujące się inwestowania w rynek hurtowni danych są świadome, 
 że głównym zadaniem zakupionego produktu nie jest genrowanie zysków tylko dostarczanie wiarygodnych i rzetelnych informacji, 
 na~podstawie których możliwe jest podjęcie decyzji strategicznych.
Jeżeli projekt hurtowni danych przechowuje niepoprawne dane, 
 bądź nie jest odpowiednio przygotowany pod danego klienta, staje się dużą stratą dla firmy.\cite{TodMan}

Głównym celem tej pracy dyplomowej jest stworzenie języka wysokiego poziomu,
 który pomoże programistom w tworzeniu procesów zasilających hurtownie danych.
 Jednym z najważniejszych założeń owego języka, jest generowanie:

\begin{itemize}
 \item szablonu pobierającego dane (źródło),
 \item szablonu pgloader lub gotowego polecenia insert,
 \item kodu umożliwiającego utworzenie tabeli,
 \item kodu języka SQL zasilającego tabele.
\end{itemize}

Pierwszy rozdział pracy opisuje hurtownie danych jej architekturę oraz przykłady wykorzystania hurtowni danych.
Rozdział drugi opisuje,
 czym są procesy zasilania hurtowni danych oraz omawia przykładowe procesy zasilania,
 które zostały realizowane w ramach niniejszej pracy.
Trzeci rozdział został w całości poświęcony teorii związanej z~tworzeniem języków interpretowanych.
Rozdział czwarty, to opis programu poparty przykładem.
Wszystkie przykłady zamieszczone w niniejszej pracy zostały przetestowane w składni języka SQL akceptowanego 
przez PostgreSQL 8.4.14 na systemie Ubuntu 10.04 LTS. 
Baza danych PstgresSQL została wybrana z następujących powodów:
\begin{itemize}
 \item bezpłatne, dobre oprogramowanie do zastosowań komercyjnych,
 \item wykorzystywane obecnie w firmie,
    w której zdobywam doświadczenie zawodowe pracując przy tworzeniu warstwy pośredniej hurtowni bazy danych.
\end{itemize}

Analizator składniowy i leksykalny został utworzony przy użyciu otwartego oprogramowania LEX i YACC, 
 które w znaczący sposób ułatwiły pracę w pierwszych dwóch etapach tworzenia języków interpretowanych i kompilatorów.
Dobre zrozumienie specyfiki działania programów Lex i Yacc, wymaga dużego zaangażowania ze strony programisty i sporego nakładu pracy.
Aby nie odbiegać od głównego celu pracy opis działania, jak również składni języka zostanie pominięty.

