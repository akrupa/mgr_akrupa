\chapter{Program}

\section{Opis implementacji programu}



Analiza leksykalna została zrealizowana, 
  przy użyciu narzędziu \textit{Lex},
  który został napisany  przez M. E. Lesk and E. Schmidt. \cite{link_lex}.
Oprogramowanie to, w znaczący sposób ułatwia tworzenie pierwszego etapu,
 przedstawionego na rysunku \ref{fig:etapy} na stronie \pageref{fig:etapy} 
 i opisanego  w podrozdziale \ref{p_leksykalna}
Poniżej został przedstawiony cały listing skryptu lex'a,
 który w całości realizuje owe zadanie.

\lstinputlisting[ label=l:lex, language=, caption = {Analiza leksykalna przy użyciu LEX'a.}] {"./skrypt/leksykalna.l"}


Tworzone symbole leksykalne są przekazywane do kolejnego etapu,
 który nazywamy analiza składniową. 
Etap ten został zrealizowany przy użyciu programu YACC,
 utworzonego przez Stephen C. Johnson.
Programy LEX i YACC, współpracują ze sobą dzięki czemu tworzenie drzewa składniowego,
 którego definicja został podana w rozdziale \ref{p_skladniowa}, 
 staje się znacznie proste.
Na początku listingu  \ref{l:yacc},
 po słowach kluczowych token zostały przedstawione symbole leksykalne,
a w dalszej części pełna gramatyka tworzonego języka.
\lstinputlisting[ label=l:yacc, language=, caption = {Analiza składniowa przy użyciu YACC.}] {"./skrypt/skladniowa.y"}  


W niniejszym programie sprawdzenie czy program buduje poprawne drzewo składniowe,
 zostało zrealizowane przy użyciu testów jednostkowych \textit{cppunit}, 
 które postały przy użyciu biblioteki \textit{log4cpp}

Program nie sprawdza poprawności atrybutów, ponieważ systemy zarządzające bazą danych sprawdzą 
 i jeżeli będą nie poprawne,
 zwrócą odpowiedni komunikat,
 a dzięki czemu program utworzony na potrzeby niniejszej pracy staje się wolny od ograniczeń związany 
 z dostępnymi typami w dowolnej bazie danych.
Odpowiedzialność za poprane wpisanie typów danych, 
 zrzucona jest na programistę,
 ponieważ to on powinien wiedzieć jakiego typu danych oczekuje 
 i czy jest obsługiwany przez dany system zarządzania bazą danych.


Ostatnim etapem jest wykorzystanie drzewa składniowego w zamierzony sposób.
W naszym programie węzłem może być:

 \begin{itemize}
  \item ciąg znaków,
  \item Klasa Tabela,
  \item Klasa Kolumna,
 \end{itemize}
które są umieszczane w klasie Wprowadzone lub od razu wykonywane są na nich działanie.

Działanie programu polega na przyjmowaniu ciągu znaków zakończonych symbolem średnika (;),
 które będziemy nazywać poleceniem, są to np:
\begin{itemize}
 \item exit; --- 
    wyjście z programu;
 \item save\_dir --- 
    nazwa katalogu do zapisu w folderze,
    w którym został uruchomiony program
 \item user\_name  login;  ---
    login użytkownika bazy użyty przez skrypt pgloadera,
 \item base\_name nazwa; ---
    nazwa bazy danych użyty przez skrypt pgloadera,
 \item pgloader nazwa --- 
    tworzenie skryptu pgloadera, nazwa odpowiada nazwie tabeli w hurtowni danych bez prefiksu inf\_, czyli przed użyciem 
    wygenerowanego skryptu musi być zdefiniowana tabela intf\_nazwa.
 \item site\_web adres\_strony  nazwa\_pliku ---
    tworzy skrypt linux o rozszerzeniu *.sh z prefiksem dane, 
    przedstawiony na listingu \ref{l:skrypt} na stronie \pageref{l:skrypt}
 \item make; ---
   Na podstawie zgromadzonych tabel wymiarów 
   i jednej tabeli faktów wygenerowanie plików sql, tworzących tabelki i kolejnych procesów zasilania 
\end{itemize}
Przesyłane są one do analizy leksykalnej i składniowej.
Wynikiem tych działań jest dodanie odpowiednich wartości do klasy Wyprowadzone 
lub odrazy wykonanie działania. 

 
\section{Przykładowe działania programu.}

\subsection{Abstrakcyjny przykład schematu gwiazdy}
Na listingu \ref{l:wy_gwiazda} został przedstawiony abstrakcyjny przykład zaprezentowany na rysunku \ref{fig:gwiazda}.
Pierwsze trzy linie zostały wyjaśnione w poprzednim podrozdziale. 
Następnie tworzone są cztery tabelki wymiarów i jedną faktową
 (kolejność nie ma znaczenia),
Zapisywane są one w klasie Wprowazdzone.

W linii 30 na omawianym listingu zostało użyte plecenie \textit{make;},
 które generuje następujące pliki:
  \begin{itemize}
   \item 
      pliki z prefiksem \textit{create\_} są to pliki tworzące tabele dla odpowiednich procesów
      dla każdej tabeli wymiarów są to pliki 
        \begin{itemize}
         \item  create\_intf\_nazwa\_wymiaru,
         \item  create\_stg\_nazwa\_wymiaru,
         \item  create\_nazwa\_wymiaru,
      w naszym przykładzie mamy 4 tabelki, więc powinniśmy otrzymać 12 plików, które tworzą dwanaście tabel
        \end{itemize}
      dla tabeli faktów dodatkowo tworzona jest jedna tabela create\_promo\_nazwa\_faktu,
      związku z tym otrzymujemy 4 pliki,
      łącznie mając ich 16. Program poinformuje nas o tworzonych plikach,
      tak jak zostało to zaprezentowane na listingu.
   \item 
      pliki, które nie zaczynają się od \textit{create\_},
      są to pliki, które są używane co cyklicznego zasilania,
      kolejność tworzenia plików nie jest przypadkowa,
      powinny one być uruchamiane w kolejność tworzenia ich,
      a jest ich łącznie 11, po dwa dla każdego wymiaru i trzy dla faktu.
  \end{itemize}


%\lstinputlisting[ label=l:we_gwiazda, caption = { Przykład  .}] {"./przyklady/we_gwiazda.txt"}
\lstinputlisting[ label=l:wy_gwiazda, caption = { Temat .}] {"./przyklady/wy_gwiazda.txt"}
\subsection{Realizacja przykładu opisanego w rozdziale \ref{r_procesy}  }

W niniejszym podrozdziale został zaprezentowana realizacja omówionego przykładu w podrozdziale \ref{p_r_analiza_procesu}.
Poprzednim podrozdziale zostało wspomniane, że kolejność tworzenia tabel nie ma znaczenia,
 jest to, spowodowane tym, że po słowach kluczowych  \textit{dimension} i  \textit{key} 
 podczas tworzenia kolumny musi wystąpić nazwa wymiaru. 
Jeżeli wymiar o tej nazwie nie zostanie zdefiniowany wcześniej to zostanie wygenerowany omówiony przykład.

W tabeli faktowej w kolumnie data\_notowania pojawiło się słowo kluczowe języka: key.
W ten sposób sygnalizujemy,
 że dana wartość jest kluczem w tabeli faktów
 i nie chcemy tworzyć dla nie klucza sztucznego. 

%\lstinputlisting[ label=l:we_np, caption = { Temat .}] {"./przyklady/we_np.txt"}
\lstinputlisting[ label=l:wy_np, caption = { Temat .}] {"./przyklady/wy_np.txt"}


Na listingu \ref{l:wy_np2} został przedstawione pełna realizacja owego przykładu.
Zostało w nim dodana tabela wymiarów,
 polecenie tworzące skrypt do pobierania
 i skrypt pgloadera.
Niestety pobierane dane do tabeli wymiarów wymagają modyfikacji,
 aby mogły być one załadowane do bazy danych przy użyciu skryptu pgloadera,
 a są nimi: 
\begin{itemize}
 \item usunięcie trzech pierwszych linii
 \item usunięcie ostatniej linii
 \item rozdzieleni kolumn znakiem 
\end{itemize}


%\lstinputlisting[ label=l:we_np2, caption = { Temat .}] {"./przyklady/we_np2.txt"}
\lstinputlisting[ label=l:wy_np2, caption = { Temat .}] {"./przyklady/wy_np2.txt"}
