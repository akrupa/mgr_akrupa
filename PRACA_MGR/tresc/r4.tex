\chapter{Program}

\section{Opis implementacji programu}



Analiza leksykalna została zrealizowana, 
  przy użyciu narzędzia \textit{Lex},
  który został stworzony przez M. E. Lesk and E. Schmidt \cite{link_lex}.
Oprogramowanie to w znaczący sposób ułatwia tworzenie pierwszego etapu analizy,
 który został przedstawiony na rysunku \ref{fig:etapy} na stronie \pageref{fig:etapy} 
 i dodatkowo opisany  w podrozdziale \ref{p_leksykalna}
Poniżej został przedstawiony listing skryptu lex'a,
 który w całości realizuje owe zadanie.

\lstinputlisting[ label=l:lex, language=, caption = {Analiza leksykalna przy użyciu LEX'a.}] {"./skrypt/leksykalna.l"}


Tworzone symbole leksykalne są przekazywane do kolejnego etapu,
 który nazywamy analiza składniową. 
Polega ona na przekształceniu symboli leksykalnych, przy użyciu gramatyki w drzewo składniowe.
Na listingu \ref{l:gramatyka} został przedstawiony opis utworzonej gramatyki  
 w rozszerzonej notacja Backusa-Naura   \ang{ EBNF, Extended Backus-Naur Form},
 w której:
\begin{itemize}
 \item symbolami terminalnymi są:
\begin{verbatim}
EXIT  SAVE_DIR TABLE_NAME BASE_NAME USER_NAME
MAKE NUMBER IDENTIFIER 
CREATE KEY FACT DIMENSION SITE_WEB SITE_WEB_ADDRESS PGLOADER
LEFTPARENTHESIS RIGHTPARENTHESIS  SEMICOLON COMMA  LEX_ERROR 
\end{verbatim}
 \item symbolami nieterminalnymi są nazwy pisane małymi literami z prefiksem y\_
\end{itemize}

\lstinputlisting[ label=l:gramatyka, language=, caption = {Gramatyka języka w notacji EBNF.}] {"./tresc/EBNF.y"}


Etap ten został zrealizowany przy użyciu programu YACC,
 utworzonego przez Stephen C. Johnson.
Programy LEX i YACC, współpracują ze sobą, dzięki czemu tworzenie drzewa składniowego,
 którego definicja został podana w rozdziale \ref{p_skladniowa}, 
 staje się znacznie prostsza.
Na początku listingu  \ref{l:yacc},
 po słowach kluczowych token zostały przedstawione symbole leksykalne,
a w dalszej części pełna gramatyka tworzonego języka.
\lstinputlisting[ label=l:yacc, language=, caption = {Gramatyka języka w składni języka YACC.}] {"./skrypt/skladniowa.y"}  


W niniejszym programie sprawdzenie poprawności budowania drzewa składniowego,
 zostało zrealizowane przy użyciu testów jednostkowych \textit{cppunit}, 
 które powstały z wykorzystaniem biblioteki \textit{log4cpp}.

Program nie sprawdza poprawności atrybutów,
 ponieważ zajmują się tym systemy zarządzające bazą danych. 
Jeśli atrybuty będą nie poprawne,
 system zwróci odpowiedni komunikat. 
Dzięki takiemu rozwiązaniu program utworzony na potrzeby niniejszej pracy
 staje się wolny od ograniczeń związanych z dostępnymi typami w dowolnej bazie danych.
Odpowiedzialność za poprawne wpisywanie typów danych, zostaje przeniesiona na programistę,
 ponieważ to on powinien wiedzieć jakiego typu danych się oczekuje 
 i czy oczekiwany typ danych jest obsługiwany przez dany system zarządzania bazą danych.

Ostatnim etapem jest wykorzystanie drzewa składniowego w określony sposób. W naszym programie węzłem może być:

 \begin{itemize}
  \item ciąg znaków,
  \item klasa Tabela,
  \item klasa Kolumna,
 \end{itemize}
które są umieszczane w klasie Wprowadzone lub od razu zostają wykonywane na nich działania.

Działanie programu polega na przyjmowaniu ciągu znaków zakończonych symbolem średnika (;), które będziemy nazywać poleceniem.
Przykładowe polecenia mogą wyglądać następująco:

\begin{itemize}
 \item exit; --- 
    wyjście z programu.
 \item save\_dir --- 
    nazwa katalogu do zapisu, w folderze w którym został uruchomiony program.
 \item user\_name login;  ---
    login użytkownika bazy danych wykorzystywany przez skrypt pgloadera.
 \item base\_name nazwa; ---
    nazwa bazy danych wykorzystywana przez skrypt pgloadera.
 \item pgloader nazwa --- 
    utworzenie skryptu pgloadera. Nazwa odpowiada nazwie tabeli w hurtowni danych bez prefiksu inf\_. Przed użyciem wygenerowanego skryptu musi być zdefiniowana tabela intf\_nazwa.
 \item site\_web adres\_strony nazwa\_pliku ---
    tworzy skrypt linux'owy o rozszerzeniu *.sh z prefiksem dane, który został przedstawiony na listingu \ref{l:skrypt}, na stronie \pageref{l:skrypt}.
 \item make; ---
   Na podstawie zgromadzonych tabel wymiarów i jednej tabeli faktów wygenerowane zostają pliki sql, odpowiedzialne za tworzenie tabeli i kolejnych procesów zasilania.
\end{itemize}

Powyższe polecenia przesyłane są do analizy leksykalnej i składniowej. 
Wynikiem tych działań jest dodanie odpowiednich wartości do klasy Wyprowadzone lub od razu wykonanie danego działania.

 
\section{Przykładowe działania programu.}  \label{p_r_przyklady}

\subsection{Abstrakcyjny przykład schematu gwiazdy}
Na listingu \ref{l:wy_gwiazda} został przedstawiony abstrakcyjny przykład,
 którego schemat widoczny jest na rysunku \ref{fig:gwiazda}.
Pierwsze trzy linie zostały wyjaśnione w poprzednim podrozdziale.
Następnie utworzone zostają cztery tabele wymiarów i jedna faktowa (kolejność tutaj nie ma znaczenia). 
Zapisywane są one w klasie Wprowazdzone.
W linii 30 na omawianym listingu zostało wykorzystane plecenie \textit{make;},
 które generuje następujące pliki:
 
  \begin{itemize}
   \item 
      pliki z prefiksem \textit{create\_} są to pliki tworzące tabele dla odpowiednich procesów.
      Dla każdej tabeli wymiarów są to pliki 
        \begin{itemize}
         \item  create\_intf\_nazwa\_wymiaru,
         \item  create\_stg\_nazwa\_wymiaru,
         \item  create\_nazwa\_wymiaru,
            w naszym przykładzie mamy 4 tabele, więc powinniśmy otrzymać 12 plików 
            odpowiedzialnych za utworzenie dwunastu tabel
        \end{itemize}
      Dla tabeli faktów dodatkowo utworzona zostaje  jedna tabela create\_promo\_nazwa\_faktu,
      związku z tym otrzymujemy 4 pliki,
      łącznie mając ich 16. Program poinformuje nas o utworzonych plikach,
      tak jak zostało to zaprezentowane na listingu.
   \item 
      pliki, które nie zaczynają się od \textit{create\_},
      są plikami, wykorzystywanymi do cyklicznego zasilania.
     Kolejność tworzenia plików nie jest tutaj przypadkowa.
     Powinny być one uruchamiane w takiej kolejności, w jakiej zostały utworzone.
     Jest ich łącznie 11, po dwa dla każdego wymiaru i trzy dla faktu.
  \end{itemize}


%\lstinputlisting[ label=l:we_gwiazda, caption = { Przykład  .}] {"./przyklady/we_gwiazda.txt"}
\lstinputlisting[ label=l:wy_gwiazda, caption = {Przykład działania programu -- Schemat gwiazdy}] {"./przyklady/wy_gwiazda.txt"}
\subsection{Realizacja przykładu opisanego w rozdziale \ref{r_procesy}  }

W niniejszym podrozdziale został zaprezentowana realizacja omówionego przykładu w podrozdziale \ref{p_r_analiza_procesu}.
Poprzednim podrozdziale zostało wspomniane, że kolejność tworzenia tabel nie ma znaczenia,
 jest to spowodowane tym, że po słowach kluczowych  \textit{dimension} i  \textit{key} 
 podczas tworzenia kolumny musi wystąpić nazwa wymiaru. 
Jeżeli wymiar o tej nazwie nie zostanie zdefiniowany wcześniej to zostanie wygenerowany omówiony przykład.

W tabeli faktowej w kolumnie data\_notowania pojawiło się słowo kluczowe języka: key.
W ten sposób sygnalizujemy,
 że dana wartość jest kluczem w tabeli faktów
 i nie chcemy tworzyć dla nie klucza sztucznego. 

%\lstinputlisting[ label=l:we_np, caption = { Temat .}] {"./przyklady/we_np.txt"}
\lstinputlisting[ label=l:wy_np, caption = {Przykład działania programu -- Notowania GPW bez tabeli wymiarów.}] {"./przyklady/wy_np.txt"}


Na listingu \ref{l:wy_np2} została przedstawiona pełna realizacja owego przykładu.
Została w nim dodana tabela wymiarów,
 polecenie tworzące skrypt do pobierania
 i skrypt pgloadera.
Niestety pobierane dane do tabeli wymiarów wymagają modyfikacji,
 aby mogły być one załadowane do bazy danych przy użyciu skryptu pgloadera.
 Modyfikacje obejmują: 
\begin{itemize}
 \item usunięcie trzech pierwszych linii,
 \item usunięcie ostatniej linii,
 \item rozdzielenie kolumn znakiem 
\end{itemize}


%\lstinputlisting[ label=l:we_np2, caption = { Temat .}] {"./przyklady/we_np2.txt"}
\lstinputlisting[ label=l:wy_np2, caption = {Przykład działania programu -- Notowania GPW.}] {"./przyklady/wy_np2.txt"}
